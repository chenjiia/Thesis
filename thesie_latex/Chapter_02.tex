% !TEX root =  ThesisMaster.tex


\chapter{Theoretical background}
	\label{CH_02}

\section{Min-entropy and information leakage in programs}
% !TEX root =  ThesisMaster.tex

\newcommand{\calC} {\mathcal{C}}


We briefly recall the mathematical theory behind computation of information leakage in programs. 
For computing information leakage, a program is usually considered as an information-theoretic channel between its inputs and outputs. Formally, 
\begin{mydef}
An information-theoretic channel $\calC$ is a triple $(S,O, C_{SO})$ such that 
\begin{itemize}
\item  $S$ is a finite set of secret input values,
\item $O$ is a finite set of secret output values, and 
\item $C_{SO}$ is a $|S| \times |O|$ non-negative matrix such that $\sum_{o_\in O} C_{SO}(s,o) =1. $
\end{itemize}
$\calC$ is said to be deterministic if for each $s\in S$ there is a unique $o \in O$  such that $C_{SO}(s,o)=1.$
\end{mydef}

For $s\in S$ and $o\in O,$ the quantity $C_{SO}(s,o)$ is  the conditional probability of obtaining output
$o$  given that the input to the channel is $s$. 
Given any information-theoretic channel $\calC=(S,O, C_{SO})$ and \emph{apriori} distribution $Prob_S$ on $S$,   we get a joint probability distribution $Prob_{S,O}$ on 
$S\times O$ given as  
$ Prob_{S,O} ((s,o))= Prob_S (s) C_{SO} (s,o).$  For the purpose of this thesis, we shall associate with each deterministic program $P$, a  deterministic  channel $P_\calC= (S,O, C_{SO})$ 
where $S$ is the set of all possible secret inputs of $P$, $O$ is the set of all possible outputs of $P$ and  $C_{SO}(s,o)=1$ iff
the program $P$ outputs $o$ on input $s.$ 

When measuring information leakage in programs using min-entropy~\cite{Smith},  an adversary  is considered which tries to guess the input to the program before and after the program is executed.  The difference in \emph{uncertainty} about  $S$ before and after the program execution is taken to be the amount of information leaked. It was proposed in~\cite{Smith}
that min-entropy be used as the measure of uncertainty. We refer the reader to~\cite{Smith} for details regarding theoretical foundations behind min-entropy.  We just point out here the relevant formulas. 

\begin{mydef}
Let $\calC=(S,O, C_{SO})$ be an information-theoretic channel and let  $Prob_S$ be an apriori distribution on $S.$
 The initial uncertainty of the adversary, written  $ H_{\infty}(S) $, is taken to be $-\log_2 \max_{s\in S} Prob_S(s) $ and final uncertainty, written  $ H_{\infty}(S | O) $,  is taken to be $-\log_2 \sum_{o\in O} \max_{s\in S} Prob_{S,O}((s,o)).$ The amount of information leaked by $\calC$, written $L_{S,O}$ is
 $$ L_{S,O} =  H_{\infty}(S) - H_{\infty}(S | O).$$
\end{mydef}

As expected, the amount of information leaked by a program is said to be the amount of information leaked by the channel $P_\calC$ associated to it.  
We are interested in computing the maximum possible amount of leakage under all possible apriori distributions (this quantity is  also known as \emph{min-capacity}). It turns out that for deterministic programs, this just amounts to computing the number  of outputs that are actually feasible, i.e., are actually realized by some input.

\begin{mythm} (\cite{Smith}) Let $\calC=(S,O, C_{SO})$ be a deterministic channel. Let 
$feasible=\set{o\in O \mathbin{|} \exists s\in S.\ C_{SO}(s,o)=1}.$  Then for any apriori distribution $Prob_S$ on $S$, we have that the information leaked
$ L_{S,O} \leq \log_2 |feasible|.$ Furthermore $ L_{S,O} = \log_2 |feasible|$ if $Prob_S$ is the uniform distribution on $S.$
\end{mythm}

\begin{remark}
Sometimes, Shannon entropy is used to measure uncertainty instead of min-entropy. However, it was shown in~\cite{Smith}, that if we measure information leaked using Shannon entropy then Shannon capacity, the maximum amount of information leaked, of deterministic programs under all possible apriori distributions  matches exactly the min-capacity. 
\end{remark}

\section{The problem and two intuitive solutions}
We define a program $P$ as follows:
\begin{mydef}
Let $bitLength\in \left\{ {0,1,2,..., 32}\right\}$ and let $P$ be a function whose input $S \in \left\{ {0,1,2,..., 1\ll bitLength-1}\right\}$ and output $O \in \left\{ {0,1,2,..., 1\ll bitLength-1}\right\}$.
\end{mydef}

And we need to count the number of outputs of $P$. We come up with two approaches: 

\begin{enumerate}
\item Put the program in a double loop and count the number of outputs. The outer loop is for possible outputs and the inner loop is for inputs. When the program in the inner loop produces an output which matches the outer loop, counter increases.
\item Let the program iterate through all input values and record the output hit results in a bit array. Counter increases when a bit flips.
\end{enumerate}

The first approach is time-consuming, while the second one is memory-consuming. We will discuss these two approaches in detail in the subsections.

\subsection{Double loop}
In Algorithm \ref{alg:doubleLoop}, for each possible output value, we iterate through the input range to see if an input can result in this output. If we hit this output, $OCounter$ increases and the code breaks out of the inner loop to continue testing the next possible output value. After the double loop finishes, the value of $OCounter$ is the number of outputs of program $P$.

\begin{algorithm}
\begin{algorithmic}
\STATE $S \leftarrow 0$
\STATE $O \leftarrow 0$
\STATE $SIn \leftarrow 0$
\STATE $OOut \leftarrow 0$
\STATE $OCounter \leftarrow 0$

\STATE $SMax \leftarrow 1 << bitLength - 1$
\STATE $OMax \leftarrow 1 << bitLength - 1$

\FOR{$O=0$ to $OMax$}
\FOR{$S=0$ to $SMax$}
\STATE $SIn \leftarrow S$ 
\STATE $OOut \leftarrow P(SIn)$ 
\COMMENT{the program $P$ takes $SIn$ as input}
\IF{$OOut = O$}
\STATE $OCounter \leftarrow OCounter + 1$
\STATE break
\ENDIF
\ENDFOR
\ENDFOR
\end{algorithmic}

\caption[Double loop]{Calculate the number of outputs using double loop.}
\label{alg:doubleLoop}
\end{algorithm}

In this approach, we declare seven variables, and all of them require $bitLength$ bits except for $OCounter$ which needs to be $bitLength + 1$ bits. The total memory usage for variables is $7 \times bitLength + 1$ at $O(bitLength)$. As with execution time, we assume program $P$ takes $t(P)$ seconds to execute, and the total execution time for the double loop when break is never reached is $2^{bitLength} \times 2^{bitLength} \times t(P)$. Thus the time complexity is $(2^{O(bitLength)}) \times t(P)$.

In order to get an estimation of how much time the double loop will take to execute, we implemented a piece of C code with an empty while loop which loops $2^{32}$ times. On our experiment PC, this loop takes on average $10.30$ seconds to complete. Were we to run a double loop in bit length of 32, the execution time would be $2^{32} \times 10.30$ seconds, which is around $1403$ years. Running the double loop at 32 bits would be infeasible.

Starting with bit length $n$, the time requirement for a full double loop is $2^{2 \times n} \times t(P)$. Increase the bit length by one and the time becomes $4 \times 2^{2 \times n} \times t(P)$, four times the previous time. Table \ref{tab:gccRun} shows the actual execution time of an empty double loop under different bit length, and the time increase follows the theoretical analysis. At bit length of 23, the execution time would exceed a day, and executing at higher bit length is impractical.

\begin{table}
\centering
\begin{tabular}{|l|l|l|}
\hline Bit length & Time(s) & Multiplier\\
\hline 14 & 0.708 & 	\\
\hline 15 & 2.797 & 3.951	\\
\hline 16 & 11.196	& 4.003	\\
\hline 17 & 44.515	& 3.976	\\
\hline 18 & 178.970	& 4.020	\\
\hline 
\end{tabular}
\caption{Execution time of an empty double loop with different bit length.}
\label{tab:gccRun}
\end{table}

\subsection{Single loop and array}
In Algorithm \ref{alg:singleLoop}, we create a bit array with size equal to the maximum number of possible outputs($1 << bitLength$, or $2^{bitLength}$) and initialize it with zeros. While we iterate through the range of $S$, we set each $OHit[P(SIn)]$ to $1$. When a $0$ turns to $1$, we increase $OCounter$. After the loop, the value of $OCounter$ is the number of outputs by program $P$. 

\renewcommand{\algorithmiccomment}[1]{// #1}
\begin{algorithm}
\begin{algorithmic}

\STATE $S \leftarrow 0$
\STATE $O \leftarrow 0$
\STATE $SIn \leftarrow 0$
\STATE $OOut \leftarrow 0$
\STATE $OCounter \leftarrow 0$

\STATE $SMax \leftarrow 1 << bitLength - 1$
\STATE $OMax \leftarrow 1 << bitLength - 1$
\STATE $OHit[OMax + 1] \leftarrow [0]$

\FOR{$S=0$ to $SMax$}
\STATE $SIn \leftarrow S$ 
\STATE $OOut \leftarrow P(SIn)$ 
\COMMENT{the program $P$ takes $SIn$ as input}
\IF{$OHit[OOut] = 0$}
\STATE $OCounter \leftarrow OCounter + 1$
\STATE $OHit[OOut] \leftarrow 1$
\ENDIF
\ENDFOR

\end{algorithmic}

\caption[Single loop]{Calculate the number of outputs using single loop and a table.}
\label{alg:singleLoop}
\end{algorithm}

In Algorithm \ref{alg:singleLoop} except for the array we have 7 variables using $7 \times bitLength + 1$ bits memory. The array $OHit[]$ is of size $2^{bitLength} \times 1$ bits making a total of $2^{bitLength} + 7 \times bitLength + 1$ bits at $2^{O(bitLength})$. As with execution time, we assume program $P$ takes $t(P)$ seconds to execute, and the execution time for the single loop is $2^{bitLength} \times t(P)$. Thus the time complexity is $(2^{O(bitLength)}) \times t(P)$.

We set the bit length to $32$. In array $OHit[]$, each element is $1$ bit and the number of elements is $2^{32}$. The total memory usage for this array is $2^{32} \times 1$ bits, which is $0.5$ gigabytes. To our knowledge, it is neither difficult nor expensive to build a PC with more than 16 gigabytes of memory, and we can get such a PC off-the-shelf from top gaming PC brands like Alienware. However, as this memory requirement grows exponentially, adding five or six bits to the bit length and the requirement will exceed the capacity of current PCs.

\subsection{Comparison}
Table \ref{tbl:execTime} and \ref{tbl:memReq} show a comparison on execution time and memory usage respectively for these two approaches. The single loop and array approach has exponential growth for both time and memory, while the double loop approach has exponential growth for time but linear growth for memory, so we choose the double loop approach. 

\begin{table}
\centering
\begin{tabular}{|l|l|l|}
\hline
 & Execution time & Growth  \\ \hline
Double loop & $2^{2 \times bitLength} \times t(P)$ & $\times 4$  \\ \hline
Single loop \& array & $2^{bitLength} \times t(P)$ & $\times 2$		\\ \hline
\end{tabular}
\caption{Comparison of the two approaches on execution time and their growth. $t(P)$ is the execution time of the program within the loop.}
\label{tbl:execTime}
\end{table}


\begin{table}
\centering
\begin{tabular}{|l|l|l|}
\hline
 & Memory requirement(bits) & Growth \\ \hline
Double loop & $7 \times bitLength + 1$ & +7 \\ \hline
Single loop \& array & $2^{bitLength} + 7 \times bitLength + 1$ & $+ 2^{bitLength} + 7$ \\ \hline
\end{tabular}
\caption{Comparison of the two approaches on memory requirement and their growth.}
\label{tbl:memReq}
\end{table}
