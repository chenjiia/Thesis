\chapter{Conclusion and future works}
	\label{CH_summary}

In this thesis we showed a novel approach of using model checking tools to compute information leakage. The principal idea is to wrap the program to be tested in a loop which counts the number of outputs it has, and instead of directly executing the whole code, we append an if statement with the counter as its condition and apply a model checking tool to check for reachability of the statement within if. We think that this approach may be faster than direct execution.

Our main work divides into three parts:
\begin{enumerate}
\item We wrote a converter that converts C-style code into boolean program, which is the input Getafix requires. Later we used the converter on all the seven test cases and it saved us a lot of time on coding the tests. 
\item We came up with an optimization method which can greatly reduce the time needed to calculate information leakage, either through model checking tools or through direct execution.
\item We tested the seven programs on both Getafix and jMoped with different bit lengths.
\end{enumerate}

We found that our approach could not run faster than direct execution. 