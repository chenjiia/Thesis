% !TEX root = ThesisMaster.tex


\chapter{Introduction}
	\label{CH_Intro}

A desirable property for a program is \emph{non-interference}~\cite{GougenMeseguer,Reynolds} which informally says that a program should  never leak  any information about its secret inputs. Formally, \emph{non-interference}~\cite{GougenMeseguer,Reynolds}
 says that the \emph{low-security observations} of the executions of a program must be independent of secret inputs. However, the desired functionality of a program usually makes non-interference unachievable. For example, a password checker behaves differently on a correct password and an incorrect password   (and its behavior even depends on the number of incorrect passwords
entered).   Therefore,  many authors~\cite{Denning,Gray,Millen,Smith} have proposed to evaluate security of programs by  quantifying the amount of information leaked. How do we measure the amount of information leaked? How to compute the information leaked?


For measuring the amount of information leaked by programs, information-theoretic measures are often used. In this approach, a program is modeled as an information channel that  {transforms} a random variable taking values from the set of confidential inputs into a random variable taking values from the set of public outputs.  Then information-theoretic measures are used to quantify the adversary's initial certainty about the secret inputs and  the  uncertainty remaining in the secret inputs after the adversary observes  the execution.
 The amount of information leaked by the program is the difference between the two. While, many information-theoretic measures can be used, it has been argued that leakage based on min-entropy is appropriate to security applications~\cite{Smith}.   Intuitively, leakage based on min-entropy measures vulnerability of the secret inputs to a \emph{single} guess of the adversary who observes the program execution.
 

Even though quantifying information leaked in programs is appealing, it is however not easy to compute information leaked in programs. Indeed it is known that the decision problem of checking whether information leaked  in non-recursive boolean programs is equal to (or less than) a given rational number is  {\PSpace}-complete~\cite{Jp2,Jp3,POST,Krish}. Recall that 
 {\PSpace} is the class of decision problems that can be solved by Turing machines that accesses at most a  polynomial number of cells on its working tape and that this class includes {\textsf{NP}} decision problems. 
 
 
 

In order to measure the information leaked by a program $P$ using min-entropy, one has to count the number of different possible outputs that may be achieved when the program is run with different inputs. The amount of information leaked is the binary logarithm of this number, 
Now, if the  input to the program $P$ consists of $n$-bits, this quantity can be computed by running the program on each of the  $2^n$ different inputs and remembering the outputs observed on each of the different inputs.  Since different inputs can lead to different outputs, this naive algorithm can take $2^n$-additional space.
This naive algorithm (which we shall call algorithm A for the rest of the section) has exponential time and exponential space complexity.  The same computation can be actually carried out in polynomial additional space by using nested iteration.
The outer iteration ranges over  all possible outputs and checks if there is an input that leads to that particular output by iterating over all possible inputs. Hence, in this alternative algorithm, the program has to be run $2^{2n}$ times.  The latter observation  immediately leads to the result that  the decision problem of checking whether information leaked  in non-recursive boolean programs is  equal to (or less than) a given rational number in {\PSpace}. Indeed, the latter algorithm (which we shall call algorithm B for the rest of the section)  provides an immediate polynomial-time reduction to the problem of checking whether the program counter reaches a given location in a program, which is also known to be {\PSpace}-complete.  


Now,  algorithm A for computing min-entropy takes above takes  at least exponential time and exponential space, while  algorithm B takes at least exponential time. Hence, they do not scale
very well  as $n$, the number of input bits increase.  However, it has been suggested in 
~\cite{POST} that one can potentially exploit the polynomial-time reduction to the reachability problem in order to estimate the amount of information leaked by using modelchecking tools as modelchecking tools were originally developed for checking whether a particular state in a program is reachable (on any possible input). These modelchecking tools do not explicitly run the program on all inputs. Instead use a variety of heuristics to solve the reachability problem. 


We tested the above  hypothesis using two popular model-checking tools, jMoped~\cite{Jmoped} and Getafix~\cite{getafix}. 
Jmoped is a tool developed for checking reachability in Java programs. The key technology used in JMoped is the use of Binary Decision Diagrams (BDDs)~\cite{lee,Bry86}. BDDs are data structures designed to efficiently   store Boolean functions. In jMoped, the input program is translated  as set of \emph{transitions} on the states of the program (a state of a non-recursive program is the current line number and the values of the variables of the program). Each transition  then models how program execution changes the state of the program.  However, instead of writing transitions explicitly,
BDDs are used to store the set of transitions.  Reachability can then be encoded as  the least fix-point solution of a set of Boolean equations which can be solved with efficient BDD operations. Getafix~\cite{getafix} is based on similar ideas except that it accepts only Boolean programs and reachability is encoded as a winning condition on a 2-player game on a  transition system. Although, these tools should work in theory, our tests for estimating min-entropy indicate that these do not scale very well  as the number of bits in the input increase. 


However, we  identified a condition under which there is a dramatic improvement in the performance of these tools. In particular, we show that if the program $P$ whose information leakage we are estimating satisfies the additional property of \emph{semi-monotonicty} then the computation of min-entropy becomes more feasible. Note that  
  if the program $P$ inputs $n$ bits and outputs $m$ bits then the program $P$ can be considered as  a function $P_{func}$ from $n$-bit binary numbers to $m$-bit binary numbers. Recall that the function $P_{func}$ is monotonically increasing  if  for each pair of $n$-bit  binary numbers $s_1, s_2$  such that  $s_1\leq s_2,$ we have that $P_{func}(s_1) \leq P_{func}(s_2).$ Note that a monotonically increasing  function we have that for each $n$-bit binary number $s$, $P_{func}(s) \geq \max_{ s'\leq s} {P_{func}(s')}.$
    We say the the program $P$ is semi-monotonically increasing if for each $n$-bit  binary number $s$ either $P_{func}(s) \geq \max_{ s'\leq s}{P_{func}(s')}$ or $P_{func}(s) \in \set{P_{func}(s') \mathbin{|} s'\leq s}.$ We can similarly define semi-monotonically decreasing programs. The program $P$ is said to be semi-monotonic if $P$ is either 
    semi-monotonically increasing or decreasing. 

The key observation that we exploit is that for semi-monotonicty we can essentially use algorithm A for computing information leakage except that we do not need to use exponential additional space. Instead, if the program $P$ is semi-monotoncially increasing (increasing respectively) then while iterating over the inputs,  we just need to remember in each iterative step  the total number of the distinct outputs seen thus far as well as the highest  (lowest respectively) output seen thus far. Thus, for semi-monotonic programs we have a new polynomial-time reduction of the decision problem of checking whether information leaked  in non-recursive boolean programs is  equal to (or less than) a given rational number to the problem of checking whether a line number in a program is reachable or not.  We now use this new reduction to estimate the information leaked in programs using jMoped and Getafix and  observe a dramatic improvement in their performance with this new reduction.



  
 
%
\paragraph{Related work.}
\label{sec:related}
The \emph{complexity} of computing the amount of leakage in Boolean programs has been   considered recently in~\cite{Jp1,Jp2,Jp3,Krish,rch:mu:fsttcs2012,Krish,POST}. 
In recent years, several automated approaches from model checking \cite{BackesKopf,Kopf,Kostas07,Kostas08}, static analysis \cite{clark94,clark05,clark07,BackesKopf}, and statistical analysis \cite{Kopf,Chothia}  have been employed to compute information leakage. We mention the most closely related work.


~\cite{Plas11, Plas13} estimates min-entropy leakage using SMT solvers. SMT solvers are tools developed to check whether a given first-order formula over a decidable first-order theory has a solution or not.  In ~\cite{Plas11, Plas13}, the authors estimate an upper bound of number of feasible outputs of a program by iterating over each pair of output bits and computing how many different values of these output bits can be achieved (i.e., how many of the values $\set{00,01,11,10}$ can be actually observed).  This technique only yields an upper bound, and it is easy to construct examples where the upper bound is a very poor estimate of the actual value.
 


Another line of closely related work is the work on computing information leakage measure using Shannon entropy~\cite{BackesKopf,Malacria07}. In this line of work, 
Shannon entropy (and not min-entropy) is used to measure uncertainty of the adversary. Usually an assumption of uniformly distributed inputs is made. When Shannon entropy is used to measure information leakage, one has to compute not only the number of feasible outputs but one also has to compute, for each feasible output,  the number of inputs that lead to that particular output. In order to compute these two things, usually an  equivalence relation on inputs is defined as follows: two inputs are equivalent if they lead to the same output. Then one needs to compute these equivalence classes. In~\cite{BackesKopf}, these equivalence classes are constructed iteratively by first starting with a single equivalence class and progressively refining them until they can be refined no further. At each iteration,   the equivalence relation is characterized using logical formulas and the refinement step uses experimental runs. In \cite{Kopf}, statistical analysis is used for this construction. In ~\cite{Malacria07}, bounded model-checking is used to compute the equivalence. In bounded model-checking, reachability is checked assuming that the program executes at most a bounded number of steps. Hence, this method gives an approximate value of information leakage. 

%While we have chosen min-entropy as the measure of information, we note that the maximum amount of information leaked by deterministic programs under all possible apriori distributions on inputs, when   matches exactly the min-capacity. 
%
%
%
%
%Introduce the reader to the current problem that you wish to solve, and why anyone should care about it.